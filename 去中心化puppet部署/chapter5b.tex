\chapter{\msyh puppet部署经验}
\begin{center}
\kai
实践出真知
\end{center}

\section{\msyh 客户端模式的puppet部署}

在数据中心里面配置puppet,通常的架构是c/s架构的部署。这种部署当然是有很多的优势。但是也不是完全完美的,也存在一些问题,例如:当客户端太多的时候,部署一台master就会不够用,执行效率是个大问题,这时候需要部署很多台master. 还有一个就是安全问题,puppet虽然使用了https的加密传输,但是这只能保证代码在传输的过程中不会被篡改。不能保证puppet manifest 本身的安全性和可靠性,假如你的master被黑,那么所有客户端就很危险。如何解决这些问题,就是后面要谈到的。\par
我在我的生产环境用了puppet的单机模式来解决这两个问题。大家知道,puppet可以直接用puppet 来执行puppet的代码。这些代码也支持模板,模块,facter变量这些,和c/s模式一样。没有区别。\par
大致的思路是,客户端通过rsync,ftp或者https等从一个中心服务器下载经过gpg签字和加密的代码到本地执行。然后把执行结果返回给puppet dashboard。这样的话,不需要维护master,能解决效率问题,不需要维护hostname,以及烦人的证书问题。另外就是发布的代码,可以用gpg进行加密和签名。
客户端在执行代码以前要对代码的签名进行验证,如果签名不对,就不执行代码。这样,即使你的中心服务器被黑,黑客也不能随意发布恶意的puppet代码给客户端,因为他不能对他发布的代码签名(因此,签名的私钥需要保存在安全的地方,比如自己的私人电脑)。我现在的做法是,在我自己的电脑上放gpg私钥,其他地方都没有,要发布puppet代码的时候,我先对代码签名,然后上传到中心服务器,下发到puppet客户端。另外的自己的pc电脑做了根目录加密,即使黑客拿到我的硬盘,也不能获取私钥。\par
\section{\msyh 具体部署}

执行gpg --gen-key 产生密钥对。
如果你linux系统开机时间不长,系统上的随机数少,产生钥匙的过程中会提示你随机数不够,需要乱敲键盘和移动鼠标来产生随机数,乱折腾下就够了。
密钥对对生成以后,需要把公钥导出,
\msyh
\begin{lstlisting}
gpg --export -a > you.pub
\end{lstlisting}
\song
并且用安全的方法拷贝到运行puppet客户端的服务器上。例如scp.
公钥拷贝到客户端以后,用
\msyh
\begin{lstlisting}
gpg --import you.pub 
\end{lstlisting}
\song
导入公钥,用root用户操作。
导入以后,可以用
\msyh
\begin{lstlisting}
gpg --list-key 
\end{lstlisting}
\song
来验证是否导入成功。
注意保障私钥的安全。
客户端有了公钥,就可以对发布的代码进行签名确认了。\par
我选择的做法是,把所有的puppet代码打包成puppet.tgz; 然后在我的个人电脑上用
\msyh
\begin{lstlisting}
gpg -a -b -s puppet.tgz 
\end{lstlisting}
\song
命令对代码进行签名,会产生一个puppet.tgz.asc的签名文件。把这两个文件上传到一个rsync服务器。然后客户端把这两个文件下载下去。然后用下面的脚本对签名进行确认和执行puppet。



\msyh
\begin{lstlisting}
rsync -avz --delete --exclude='.svn' puppet@www.example.com::puppet/ /opt/puppet/
cd /opt/puppet/
gpg --verify puppet.tgz.asc
[ $? -ne 0 ]&&exit 0
mkdir /tmp/xy-puppet
tar zxf puppet.tgz -C /tmp/xy-puppet
rsync -avz --delete /tmp/xy-puppet/puppet/ /etc/puppet/
puppet /etc/puppet/manifests/site.pp

\end{lstlisting}
\song

