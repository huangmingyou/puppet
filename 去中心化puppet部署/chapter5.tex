\chapter{\msyh puppet高级内容}
\begin{center}
\kai
欲穷千里目,更上一层楼
\end{center}
\par
puppet在大规模的生成环境中,如果只有一台puppetmaster,会忙不过来的,因为puppet是用 ruby写的,ruby是解析型语言,每个客户端来访问,都要解析一次,当客户端多了就忙不过来,所以需要扩展成一个服务器组。puppetmaster可以看作一个web服务器,实际上也是由ruby提供的web服务器模块来做的。因此可以利用web代理软件来配合puppetmaster做集群设置。这方面的资料在官方网站有详细介绍,例如puppet+nginx等等。\par
puppet后台运行的时候,默认是半小时执行一次,不是很方便修改。可以考虑让 puppetd 不运行在后台,而使用crontab来调用,执行完毕就退出,这样可以精确的控制所有的puppetd客户端的执行时间,分散执行时间也可以减轻puppetmaster服务器的压力。\par
puppet还支持外部资源,所谓外部资源,就是发布在客户端以外的资源,所有客户端都可以共享这些资源。\par
最后来看看puppet的工作细节,分为下面几个步骤:\par
一.客户端puppetd 调用facter, facter探测出主机的一些变量,例如主机名,内存大小,ip地址等。 pupppetd 把这些信息通过ssl连接发送到服务器端。\par
二.服务器端的puppetmaster 检测客户端的主机名,然后找到 manifest里面对应的 node 配置, 然后对该部分内容进行解析,facter送过来的信息可以作为变量处理,node牵涉到的代码才解析,其他没牵涉的代码不解析。 解析分为几个阶段,语法检查,如果语法错误就报错。\par
如果语法没错,就继续解析,解析的结果生成一个中间的“伪代码”,然后把伪代码发给客户端。\par
三.客户端接收到“伪代码”,并且执行,客户端把执行结果发送给服务器。\par
四.务器端把客户端的执行结果写入日志。\par

